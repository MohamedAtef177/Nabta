\chapter{Conclusions \& Future Work}

\section{Conclusions}

There are many developed methods for the detection and classification 
of plant diseases using the leaves of plants. However, there is still 
no efficient and effective commercial solution that can be used to
identify diseases and plants identification. NABTA used (InceptionV3)
for the detection of plant identification and plant diseases using healthy- and
diseased-leaf images of plants. \\

To train and test the model, we used the standard PlantVillage dataset with
54,303 images, which were all captured in laboratory conditions. This dataset
consists of 38 classes of different plants, we used a dataset with 4800 images 
for cherry, 4800 for peach, 4800 for Pepper, 7200 for Potato, 4800 for Strawberry, 
7440 for Apple, 9600 for Grape, 9600 for Corn, after splitting the
dataset into 80 -- 20 (80\% for training, 20\% for testing),
NABTA achieved the best accuracy rate of \\[4pt]
\indent \textbf{99.52\%} in the InceptionV3 model for apple, \\[4pt]
\indent \textbf{99.9\%} in the InceptionV3 model for cherry, \\[4pt]
\indent \textbf{99.27\%} in the InceptionV3 model for grape, \\[4pt]
\indent \textbf{99.4\%} in the InceptionV3 model for peach, \\[4pt]
\indent \textbf{98.28\%} in the InceptionV3 model for corn, \\[4pt]
\indent \textbf{99\%} in InceptionV3 model for potato, \\[4pt]
\indent \textbf{99.4\%} in the InceptionV3 model for Pepper, \\[4pt]
\indent and \textbf{99.9\%} in the InceptionV3 model for strawberries. \\[4pt]
Moreover, NABTA allows the user to search for a plant and know information 
about it by entering the name of the plant in the field 
search bar and the data for this plant will be displayed. \\

Finally, NABTA can classify the ripeness stages of strawberry. The proposed 
system has three main stages; pre-processing, feature extraction, 
and ripeness classification. which uses Object detection to allow us 
to identify and locate objects in an image and determine and track their 
precise locations, all while accurately labeling them, The proposed classification 
approach was implemented by applying to resize and extracting color components 
for each image. Then, feature extraction was applied to each pre-processed image, 
this dataset consists of 4 classes of different stages of ripeness 
(Green, white, pink, and green) Yolov5 model is developed for ripeness
stage classification. For Strawberry ripeness assessment, based on the obtained
experimental results, the highest ripeness mAP of 91.2\%.

\section{For Future work}
We aim to:
\begin{enumerate}
  \item Increase the number of plants in our application.
  \item Inserting the robotics part for automatic plant care.
  \item Detection and estimation of weeds.
  \item Suggesting Fertilizers, Herbicides, and Insecticides for early detection of the disease for speedy treatment.
  \item Follow the stages of plant growth.
  \item Add an export user to manage data and model.
  \item Add System to care with plants.
\end{enumerate}
